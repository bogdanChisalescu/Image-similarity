\section{Performante}

\quad Pentru a analiza performanta si pentru a alege varianta de LBP
optima va trebui sa definim o metrica sau un mod prin care se va putea
discerne intre rezultatele obtinute. Avand in vedere faptul ca, algoritmul
determina si alege cele mai similare imagini dintr-un mediu controlat
in totalitate de utilizator, rezultatele vor fi intr-o mare masura
dependente de setul de date, adica de baza de imagini pusa la dispozitie.
In acest sens am considerat ca pot aparea doua tipare distincte destul 
de probabile in setul de date: o situatie in care imaginile apartin mai multor clase si exista mai mult de o imagine din fiecare clasa si o 
situatie in care imaginile apartin mai multor clase dar nu exista mai mult de o imagine din fiecare clasa. Pentru primul caz am definit 
criteriul de performanta asemanator cu modul in care un subiect uman
ar alege cele mai similare imagini, deci in cazul in care exista o imagine
sursa din clasa $C$ ar fi alese ca cele mai similare tot imagini 
apartinand clasei $C$; astfel am impartit setul de date in 5 clase de
imagini: {monumente, dinozauri, flori, cai si masini}. Alegandu-se
o imagine din aceste 5 clase ca imagine sursa, vor fi considerate ca
rezultate valide doar imaginile din aceeasi clasa cu aceasta, altfel orice alta imagine dintr-o clasa distincta va fi considerata ca identificata gresit. In acest mod scorul maxim pentru o imagine sursa
este de 3/3 imagini identificate corect.
\quad Pentru al doilea caz, in care nu exista mai multe imagini din aceeasi clasa, nu am putut defini o metrica deoarece criteriul uman de
selectie al imaginilor similare in acest caz este mult prea subiectiv.
Am considerat pentru o mai buna aproximare scorul mediu pe intreg setul de date ca indice al performantei, deoarece ne intereseaza performantele
algoritmului aplicat pe un numar cat mai mare de clase.

\begin{table}[H]
	\centering
	\begin{tabular}{|l|l|l|}
		\hline
		Tipul de LBP        &  & Scorul mediu \\ \hline
		&  &              \\ \hline
		LBP fundamental     &  & 1.56         \\ \hline
		LBP mediu           &  & 1.14         \\ \hline
		LBP circular (8,1)  &  & 1.26         \\ \hline
		LBP circular (16,2) &  & 1.08         \\ \hline
		LBP circular (8,3)  &  & 0.9          \\ \hline
	\end{tabular}
	\caption{Scorul mediu pentru fiecare varianta de LBP}
\end{table}


\quad Din datele obtinute putem observa ca cea mai buna performanta o are varianta de LBP fundamental cu 1.56 imagini identificate corect in medie pentru o anumita imagine sursa.