\section{Interfata grafica}

\quad La proiectarea interfetei grafice au fost luate in considerare notiunile
teoretice cu privire la dezvoltarea interfetelor vizuale, prezentate in cursul
2 al disciplinei Interfete Om-Masina. Au fost respectate in acest mod: simetria,
echilibrul, regularitatea, secventialitatea, unitatea, proportionalitatea,
si gruparea. \\
Atunci cand este lansata in executie aplicatia, utilizatorul este intampinat de 
primul stadiu al interfetei grafice.

\begin{figure}[H]
	\includegraphics[width=14cm]{interface1.png}
	\centering
	\caption{Primul stadiu al interfetei grafice}
\end{figure}

Pentru a respecta principiul secventialitaii, utilizatorul are la dispozitie 
performarea unei singure actiuni si anume incarcarea unei imagini, imagine ce 
reprezinta imaginea sursa pentru care se vor identifica mai tarziu cele 3 alte
imagini similare. Deasupra butonului este plasat un text informational ce arata
starea curenta a aplicatiei sau actiunea ce este curent intretinuta, de asemenea functionalitatea butonului este inscrisa pe suprafata sa.


La apasarea butonului, o fereastra de explorare a fisierelor se va deschide si va fi necesara selectarea imaginii sursa. Dupa ce este selectata, imaginea se va incarca in fereastra
interfetei.

\begin{figure}[H]
	\includegraphics[width=14cm]{interface2.png}
	\centering
	\caption{Stadiul intermediar al interfetei grafice}
\end{figure}

In stadiul intermediar este prezentata imaginea sursa incarcata, si in textul 
informational este printat numele acestei imagini asa cum apare in fisier. 
Tot in aceasta etapa, respectand principiul gruparii, sub primul buton este afisat
un altul ce are ca functionalitate pornirea algoritmului, ce consta in determinarea
setului de trasaturi ale tuturor imaginilor ce se afla in acelasi director cu imagnea
sursa. Dupa calcularea descriptorilor, pe baza distantei euclidiene sunt selectate
cele mai similare 3 imagini si afisate conform principiului regularitatii sub imaginea
sursa. Interfata a fost proiectata cu ajutorul modulului wx. Plecand de la structura
clasica a unei implementari de interfata wxPython (clasa Frame, clasa Panel) au fost
introduse functionalitatile dorite, si anume: butoanele proiectate folosind constructorul
wx.Button(...), textul folosind wx.StaticText(...), atasarea unor evenimente butoanelor
cu metoda wx.Bind(...) si afisarea imaginilor folosind metoda static.Bitmap(...).

\begin{figure}[H]
	\includegraphics[width=14cm]{interface3.png}
	\centering
	\caption{Stadiul final al interfetei grafice}
\end{figure}

In stadiul final, interfata da posibilitatea utilizatorului de a alege o alta imagine
sursa, dar si de a calcula din nou descriptorii LBP si distantele aferente, cat si selectia
celor mai similare 3 imagini, insa datorita faptului ca in algoritmul de determinare
al imaginilor similare si al descriptorilor nu intervin procese stochastice rezultatele vor
fi aceleasi pentru aceeasi imagine sursa.



