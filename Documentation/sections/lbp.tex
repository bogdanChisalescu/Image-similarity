\section{LBP si distanta euclidiana}

\subsection{Local binary pattern}

\quad In continuare vom descrie metodele LBP ce au fost folosite in
actuala lucrare. Toate variantele de LBP prezentate opereaza pe imagini 
cu nivele de gri, in cazul in care imaginea de prelucrat nu este o imagine
exclusiv cu nivele de gri, aceasta va fi convertita.

\quad Local binary pattern sau in forma sa abreviata LBP este o operatie punctuala centrata  
pe o vecinatate a unei imagini in urma careia se poate forma un descriptor vizual. Aceasta 
operatie este caracterizata de extragerea unei texturi.


\subsection{LBP fundamental}

\quad Operatia LBP fundamentala presupune parcurgerea imaginii pixel cu pixel si extragerea 
unei vecinatati 
\[
V_8 = \{ (1,-1), (0,1), (1,1), (0,-1), (0,0), (0,1),(-1,-1), (-1,0), (-1,1) \}	
\]

cu coordonate relative fata de valoarea curent prelurata. Pixelul curent aflat in prelucrare ce va avea in vecinatate coordonatele 
$(0,0)$ este considerat ca valoare de prag si sunt comparati cu acesta
ceilalti pixeli apartinand vecinatatii. In urma comparatiei, daca valoarea din vecinatate este mai mare decat valoarea pixelului central 
atunci aceasta ia valoarea 1, altfel ii este asignata valoarea 0. In
acest fel se creeaza un tipar de 1 si 0. Acest tipar este parcurs in sens orar in vecinatate, formadu-se cu el un numar binar ce mai apoi convertit in numar zecimal ne da valoarea descriptorului pentru acea 
vecinatate. \\
\null \quad Aplicand LBP pe dimensiunea intregii imagini obtinem un descriptor
de aceleasi dimensiuni cu imaginea initiala. In prezenta lucrare 
imaginile sunt parcurse astfel incat valorile marginale ce depasesc
limitele imaginii sunt ignorate prin pozitionarea convenabila a 
punctului de start. \\

\begin{figure}[H]
	\includegraphics[width=17cm]{basic_lbp.png}
	\centering
	\caption{Schema de aplicare LBP fundamental pe vecinatatea $V_8$}
\end{figure}


\subsection{LBP-mediu}

\quad Aceasta forma de LBP este identica cu cea fundamentala cu exceptia
faptului ca, valoarea de prag este considerata media tuturor pixelilor
ce apartin vecinatatii $V_8$.


\subsection{LBP circular}

\quad LBP-ul circular are la baza acelasi principiu ca si cel
fundamental. Consideram imaginea de prelucrat $I(x,y)$ si coordonatele
punctului curent relative la imagine $x_p$ si $y_p$. Vecinatatea 
circulara este definita de cercul $C(I(x_p,y_p), R)$. Pe acest 
cerc sunt alese un numar de P puncte echidistante. Coordonatele 
punctelor sunt calculate dupa urmataorele formule:
	
\begin{equation}
	x_k = x_p + Rcos(\frac{2 \pi k}{P}), k \in {\{0,1,...,P-1\}}
\end{equation}

\begin{equation}
	y_k = y_p - Rsin(\frac{2 \pi k}{P}), k \in {\{0,1,...,P-1\}}
\end{equation}

In cazul in care $x_k$ si $y_k$ $\notin \mathbb{Z}$ si deci nu se 
poate obtine direct $ I(x_k, y_k) $, aceasta va fi interpolata 
biliniar. \\

Consideram un semnal bidimensional discret $ f(x,y) $, si 2 
indici $x_k$ si $y_k$ pentru care semnalul $ f $ nu este
definit. Putem interpola biliniar valoarea semnalului $ f $
in punctul $(x_k ,y_k)$ considerand cele mai apropiate 4 puncte ale semnalului
de $(x_k ,y_k)$. Notand indicii acestor puncte considerandu-l pe
cel din stanga jos origine, putem afla valoarea semnalului 
$f(x_k ,y_k)$ ca fiind: 

\begin{equation}
	f(x_k ,y_k) \approxeq f(0,0)(1-x)(1-y) + f(1,0)x(1-y) + f(0,1)(1-x)y
	+ f(1,1)xy
\end{equation}

\begin{figure}[H]
	\includegraphics[width=2.2cm]{visual_interpolation.png}
	\centering
	\caption{Interpretare vizuala a interpolarii biliniare}
\end{figure}

\newpage

Avand la dispozitie valoarea intensitatii imaginii in punctul central 
$ I(x_p,y_p) $ si valorile asociate punctelor echidistante 
$ I(x_k, y_k) , k \in {\{0,1,...,P-1\}} $ putem determina valoarea 
returnata de LBP circular pentru punctul $ (x_p, y_p) $
ca fiind: \\

\begin{equation}
	LBP_{(P,R)} = \sum_{k=0}^{P-1}{S(I(x_k, y_k) - I(x_p,y_p))2^{k}}
\end{equation}


unde $ S(u) $ este definita ca: 

\begin{equation}
	S(u)= \left\{
	\begin{array}{ll}
		1, & u\geq 0 \\
		0, & u< 0 \\
	\end{array} 
	\right.
\end{equation}


\begin{figure}[H]
	\includegraphics[width=7cm]{lbp_circular.png}
	\centering
	\caption{Vecinatatile circulare (8,1), (16,2), respectiv (8,3)}
\end{figure}


\begin{figure}[H]
	\centering
	\begin{tabular}{cc}
		\subfloat[Imaginea originala]{\includegraphics[width = 4.3cm]{orig.png}} &
		\subfloat[LBP fundamental]   {\includegraphics[width = 4.3cm]{basic.png}}\\
		\subfloat[LBP mediu]         {\includegraphics[width = 4.3cm]{mean.png}} &
		\subfloat[LBP circular]      {\includegraphics[width = 4.3cm]{circular.png}}\\
	\end{tabular}
	\caption{Descriptorii LBP}
\end{figure}




\subsection{Distanta euclidiana}


